\documentclass[unicode,10pt]{beamer}
\setbeamertemplate{caption}[numbered]
\usepackage[font=small,format=plain,labelfont=bf,up,textfont=it,up]{caption}
\usepackage[utf8]{inputenc}
\usepackage[english,russian]{babel}
\usepackage{url}
\usepackage{hyperref}
\usepackage[noend]{algorithmic}
\usepackage{algorithm}


\usetheme{AnnArbor}
\usecolortheme{crane}
\graphicspath{{pics/}}
\definecolor{blue}{rgb}{0.00,0.00,1.00}
\newcommand{\bi}{\begin{itemize}}
\newcommand{\ei}{\end{itemize}}
\newcommand{\be}{\begin{enumerate}}
\newcommand{\ee}{\end{enumerate}}


%\usepackage{pscyr}              % шрифты
\def\tmdefault{fsv}
\input glyphtounicode.tex       % для поиска и копирования
\pdfgentounicode=1

\title[ML\&DM]{Программирование алгоритмов на С++}
\author[]{Кашницкий Ю.}

\begin{document}

\section{Программирование алгоритмов на С++}

\begin{frame}{Занятие 1}
\begin{block}{Задача 1.1}
Найти максимальное расстояние (модуль разности) между чётными числами в последовательности, так что оба числа окружены нечётными
 (если крайний элемент, то проверяем только одного соседа).
\end{block}

\begin{block}{Задача 1.2}
Вычислите, сколько полей на шахматной доске могут быть конечной точкой пути коня за M ходов из заданной точки (её можно задавать случайно).
\end{block}

\begin{block}{Задача 1.3}
Сгенерируйте $M$ случайных точек на единичной сфере в пространстве относительно равномерного распределения. 
Вычислите, сколько из них находятся на расстоянии $ < a ( = 0.1 )$ от треугольника, построенного на каких-либо трёх других точках. 
Как оценивается сложность вашего алгоритма? 
\end{block}

\end{frame}
\begin{frame}{Занятие 2}

\begin{block}{Задача 1.7}
Есть большая прямоугольная таблица символов, могут встречаться: \# и пробел . 
Пустые области - это окна. Написать алгоритм, который максимально быстро определяет, являются ли все окна прямоугольными. Какова его сложность? 
\end{block}

\begin{block}{Задача 1.7.2}
В условиях предыдущей задачи посчитать количество областей связности. 
\end{block}



\end{frame}
\end{document}
