\documentclass{report}
\usepackage[utf8]{inputenc}
\usepackage[english,russian]{babel}
\usepackage{amsmath}
\usepackage{pscyr}

\begin{document}
\thispagestyle{empty}
\begin{center}
\Large
\textbf{ШАД. Первый этап отбора. Вариант 1.}
\end{center}
	
\vspace{1 cm}
\Large			
						
\begin{enumerate}
\item \textbf{Предел последовательности}\\
 $$\lim_{n\to\infty} \left|\frac{1}{n} -  \frac{2}{n} + \frac{3}{n} - \ldots + \frac{(-1)^{n-1}n}{n} \right|$$

\item \textbf{Предел функции} \\
$$\lim_{x\to 3} \frac{\sqrt{x+6} - \sqrt{10x-21}}{5x-15}$$

\item \textbf{Интеграл} \\
$$\int_{1}^{e^2} \frac{ln^2x}{\sqrt{x}} dx$$

\item \textbf{Собственный вектор} \\
\[ \left( \begin{matrix}
2 & -5 & -8 \\
-8 & -6 & -2 \\
4 & 3 & 1 \end{matrix} \right)\] Найти собственный вектор, соответствующий максимальному собственному значению. 

\item \textbf{Матрица с параметром} \\
\[ \left( \begin{matrix}
a-6 & 1 & 1 \\
-8 & a & 2 \\
4-6a & a & a^2+a-4 \end{matrix} \right)\] При каких значениях $a$ ранг матрицы равен 2?

\item \textbf{Подстановки} \\
 Найти количество инверсий подстановки $X$, определяемой из равенства $AXB = C$, где \\
A =  $\left( \begin{matrix} 1 & 2 & 3 & 4 & 5 & 6 & 7 \\ 7 & 3 & 2 & 1 & 6 & 5 & 4 \end{matrix} \right)$ , 
B =   $\left( \begin{matrix} 1 & 2 & 3 & 4 & 5 & 6 & 7 \\ 3 & 1 & 2 & 7 & 4 & 5 & 6 \end{matrix} \right)$,
C =   $\left( \begin{matrix}1 & 2 & 3 & 4 & 5 & 6 & 7 \\ 5 & 1 & 3 & 6 & 4 & 7 & 2 \end{matrix} \right)$.

Подстановки применяются справа налево.

\item \textbf{Вагончики} \\
 У Стаса есть три красных, четыре синих и пять зеленых вагончиков, из которых ему хочется собрать паровозик. Сколько есть способов сделать это так, чтобы все вагончики были использованы и никакие два синих вагончика не стояли рядом?

\item \textbf{Школьники} \\
Мимо киоска с мороженым каждый день ходят школьники, причем $\frac{4}{5}$ из них учатся в лицее, а остальные -- в гимназии. В среднем  $1\%$ лицеистов и $7\%$ гимназистов останавливаются и покупают мороженое. Какова вероятность того, что проходящий мимо школьник купит мороженое?

\item \textbf{Переворот} \\
В строке $S$ выделили подстроку, состоящую из символов с $i$-го по $j$-й включительно (символы строки $S$ нумеруются с единицы) и поменяли местами $i$-й символ с $j$-м, $(i+1)$-й с $(j-1)$-м и так далее (перевернули подстроку). Выведите строку $S$ после внесенных изменений. \\
\newpage
\textbf{Формат ввода}: В первой строке входного файла содержится строка $S$, длиной не более 1000 символов, во второй – числа $i$ и $j\ (i \leq j)$.\\
\textbf{Формат вывода}: В выходной файл выведите ответ на задачу.\\
\textbf{Пример}:
\[ \begin{array}{lr}
\mbox{\textbf{Ввод}} & \mbox{\textbf{Вывод}} \\
\mbox{vjhoamkts} & \mbox{vjhoamtks}  \\
\mbox{7 8} & \\
 \end{array}\]
\[ \begin{array}{ll}
\mbox{Ограничение времени} & \mbox{1 секунда} \\
\mbox{Ограничение памяти} &	\mbox{64Mb} \\
\mbox{Ввод} &	\mbox{стандартный ввод или input.txt} \\
\mbox{Вывод} &	\mbox{стандартный вывод или output.txt} \\
 \end{array}\]

\item \textbf{Результаты олимпиады} \\
Во время проведения олимпиады каждый из участников получил свой идентификационный номер -- натуральное число. Необходимо отсортировать список участников олимпиады по количеству набранных ими баллов.\\
\textbf{Формат ввода}: На первой строке дано число $N\ (1 \leq N \leq 1000)$ — количество участников. На каждой следующей строке даны идентификационный номер и набранное число баллов соответствующего участника. Все числа во входном файле не превышают $10^5$.\\
\textbf{Формат вывода}: В выходной файл выведите исходный список в порядке убывания баллов. Если у некоторых участников одинаковые баллы, то их между собой нужно упорядочить в порядке возрастания идентификационного номера.\\
\textbf{Пример}:
\[ \begin{array}{lr}
\mbox{\textbf{Ввод}} & \mbox{\textbf{Вывод}} \\
\mbox{3} & \mbox{305 90}  \\
\mbox{101 80} &  \mbox{101 80} \\
\mbox{305 90} &  \mbox{200 14} \\
\mbox{200 14} &  \\
 \end{array}\]
\[ \begin{array}{ll}
\mbox{Ограничение времени} & \mbox{1 секунда} \\
\mbox{Ограничение памяти} &	\mbox{64Mb} \\
\mbox{Ввод} &	\mbox{стандартный ввод или input.txt} \\
\mbox{Вывод} &	\mbox{стандартный вывод или output.txt} \\
 \end{array}\]
\end{enumerate}

\end{document}