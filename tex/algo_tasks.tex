\documentclass{article}
\usepackage[utf8]{inputenc}
\usepackage[T2A]{fontenc} 
\usepackage[english,russian]{babel}
\usepackage{amsmath}
\usepackage[hmarginratio=1:1,top=32mm,columnsep=20pt]{geometry} % Document margins
%\usepackage{pscyr}

\begin{document}
\thispagestyle{empty}
\begin{center}
\Large
\textbf{Программирование алгоритмов на С++. Задачи}
\end{center}
	
\vspace{1 cm}
\Large			
						
\begin{itemize}

\item \textbf{Задача 1} \\
Найти максимальное расстояние (модуль разности) между чётными числами в последовательности, так что оба числа окружены нечётными (если крайний элемент, то проверяем только одного соседа).

\item \textbf{Задача 2. Ходы коня} \\
Вычислите, сколько полей на шахматной доске могут быть конечной точкой пути коня за M ходов из заданной точки (её можно задавать случайно).

\item \textbf{Задача 3. Случайные точки на сфере} \\
Сгенерируйте $M$ случайных точек на единичной сфере в пространстве относительно равномерного распределения. 
Вычислите, сколько из них находятся на расстоянии $ < a ( = 0.1 )$ от треугольника, построенного на каких-либо трёх других точках. 
Как оценивается сложность вашего алгоритма? 

\item \textbf{Задача 4. Максимум справа за один проход} \\
Дан массив $a$ длины $n$. Найдите за один проход по массиву \\$max_j \sum_{k=j}^na[k]$. Ограничение по памяти -- $O(1)$.

\item \textbf{Задача 5. Максимум максимумов} \\
Дан массив $a$ длины $n$  без нулей. Для $k$-ого подмассива с элементами одного знака (пусть его начальный и конечный индексы -- $i_1$ и $i_2$)
обозначим $m_k = max_j\{S_j^\prime, S_j^{\prime\prime}\}$, где $S_j^\prime = \sum_{r=i_1}^j(x_r mod\ 5 - 2)$, a $S_j^{\prime\prime} = \sum_{r=j}^{i_2}(x_r mod\ 7 - 3)$ . 
Найти $max_k\{m_k\}$
Ограничение по времени -- $O(n)$, по памяти -- $O(1)$.

\item \textbf{Задача 6. Минимум максимумов.} \\
Дан массив $a$ длины $n$  без нулей. Для $k$-ого подмассива с элементами одного знака (пусть его начальный и конечный индексы -- $i_1$ и $i_2$)
обозначим $m_k = max_j\{S_j^\prime, S_j^{\prime\prime}\}$, где $S_j^\prime = \sum_{r=i_1}^jx_r$, a $S_j^{\prime\prime} = \sum_{r=j}^{i_2}x_r$ . 
Найти $min_k\{m_k\}$
Ограничение по времени -- $O(n)$, по памяти -- $O(1)$.

\item \textbf{Задача 7. Окна} \\
Есть большая прямоугольная таблица символов, могут встречаться: '\#' и пробел. 
Пустые области - это окна. Написать алгоритм, который максимально быстро определяет, являются ли все окна прямоугольными. Какова его сложность? 

\item \textbf{Задача 8. Лабиринт} \\
Есть большая прямоугольная таблица символов, могут встречаться: '\#' и пробел. '\#' соответствует стенкам лабиринта, пробелы - дорожкам. В левой нижней и правой верхней позициях пробелы.
Написать алгоритм, который определяет, есть ли путь из левого нижнего угла в правый верхний.
\end{itemize}

\end{document}
