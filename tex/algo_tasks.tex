\documentclass[unicode,10pt]{beamer}
\setbeamertemplate{caption}[numbered]
\usepackage[font=small,format=plain,labelfont=bf,up,textfont=it,up]{caption}
\usepackage[utf8]{inputenc}
\usepackage[english,russian]{babel}
\usepackage{url}
\usepackage{hyperref}
\usepackage[noend]{algorithmic}
\usepackage{algorithm}


\usetheme{AnnArbor}
\usecolortheme{crane}
\graphicspath{{pics/}}
\definecolor{blue}{rgb}{0.00,0.00,1.00}
\newcommand{\bi}{\begin{itemize}}
\newcommand{\ei}{\end{itemize}}
\newcommand{\be}{\begin{enumerate}}
\newcommand{\ee}{\end{enumerate}}


%\usepackage{pscyr}              % шрифты
\def\tmdefault{fsv}
\input glyphtounicode.tex       % для поиска и копирования
\pdfgentounicode=1

\title[ML\&DM]{Программирование алгоритмов на С++}
\author[]{Кашницкий Ю.}

\begin{document}

\section{Программирование алгоритмов на С++}
\begin{frame}{Занятие 1}

\begin{block}{Задача 1.4}
Дан массив $a$ длины $n$. Найдите за один проход по массиву $max_j \sum_{k=j}^na[k]$. Ограничение по памяти -- $O(1)$.
\end{block}

\begin{block}{Задача 1.5}
Дан массив $a$ длины $n$  без нулей. Для $k$-ого подмассива с элементами одного знака (пусть его начальный и конечный индексы -- $i_1$ и $i_2$)
обозначим $m_k = max_j\{S_j^\prime, S_j^{\prime\prime}\}$, где $S_j^\prime = \sum_{r=i_1}^j(x_r mod\ 5 - 2)$, a $S_j^{\prime\prime} = \sum_{r=j}^{i_2}(x_r mod\ 7 - 3)$ . 
Найти $max_k\{m_k\}$
Ограничение по времени -- $O(n)$, по памяти -- $O(1)$.
\end{block}

\begin{block}{Задача 1.6}
Дан массив $a$ длины $n$  без нулей. Для $k$-ого подмассива с элементами одного знака (пусть его начальный и конечный индексы -- $i_1$ и $i_2$)
обозначим $m_k = max_j\{S_j^\prime, S_j^{\prime\prime}\}$, где $S_j^\prime = \sum_{r=i_1}^jx_r$, a $S_j^{\prime\prime} = \sum_{r=j}^{i_2}x_r$ . 
Найти $min_k\{m_k\}$
Ограничение по времени -- $O(n)$, по памяти -- $O(1)$.
\end{block}

\end{frame}
\end{document}
